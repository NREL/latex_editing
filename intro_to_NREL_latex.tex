% ---------------
% PREAMBLE
% ---------------
\newif\iflatextortf

\iflatextortf
 % tell latex2tortf if this is an article or report
 \documentclass[12pt,letterpaper]{report}
 \input{NRELLatex2rtf.tex}
\else
 \documentclass[report]{nrel} 
\fi

% -----------
% MACRO
%------------
\newcommand{\filetype}[1]{#1}

% -----------------------------------
% DOCUMENT PROPERTIES
% -----------------------------------
\title{Writing NREL documents using \LaTeX}
\author{A. Clifton, A. Platt, P. Fleming, M. Lawson}

% -------------------------------------
% DOCUMENT STARTS HERE
% -------------------------------------
\begin{document}
\maketitle
\pagenumbering{roman}
\setcounter{page}{3}

\chapter*{Abstract}
\LaTeX\ is a document preparation system that is very popular for preparing technical documents. However, it is \emph{not} WYSIWYG and lacks some of the reviewing and editing tools of other word processing software. For this reason special care has to be taken when preparing NREL documents in \LaTeX. This document serves as a template and combines a short introduction to \LaTeX\ with details of how to implement NREL's style and formatting guidelines.

\clearpage
\renewcommand{\contentsname}{Table of Contents}
\tableofcontents
\clearpage
\listoffigures
\listoftables
\clearpage
\pagenumbering{arabic}


% ---------------------------------------
% CHAPTER 1. LATEX AT NREL
% ---------------------------------------
\chapter{Requirements for NREL documents}

There are well-defined requirements for all documents that are published by NREL. 

\section{Use of english and other conventions}
The NREL in-house style is described at \url{http://www.nrel.gov/extranet/communications/styleguide.html}. This details the conventions that should be used when writing NREL documents.

\section{Formatting}
NREL publishes templates for reports and other technical documents. These are designed to be used with most common WYSIWYG programs and \LaTeX{}. Templates are posted online at \url{http://www.nrel.gov/extranet/communications/report_template.html} and updated regularly. 

\chapter{How to Make \LaTeX\ documents that meet NREL's requirements}

\section{What is \LaTeX?}
\label{sec:examples}

\LaTeX\ is a mark-up language that describes how a document should be prepared. Three things are needed to make a \LaTeX\ document:
\begin{enumerate}
\item A source document, usually with extension \emph{.tex}
\item Some packages and classes that help turn what's in the source document into something helpful
\item A compiler, also referred to as a working \LaTeX\ installation.
\end{enumerate}

At first glance the source document looks like a programming language, and that's because it is: \LaTeX\ is not WYSIWYG, like many of the document preparation tools in common use today. A good analogy is html.

The wikibook at \url{http://en.wikibooks.org/wiki/LaTeX} is an excellent resource. There are also several internet forums such as \url{tex.stackexchange.com} that may be useful.

\section{General Process}
An outline of the process for producing NREL documents using \LaTeX{} is given in Table \ref{Tab:NRELprocess}. Please note that this process is subject to revision without warning.

\begin{table}[!h]
\centering
\caption{NREL's process for producing and reviewing \LaTeX{} files}
\label{Tab:NRELprocess}
\begin{tabular*}{\textwidth}{llp{0.5\textwidth}r}
\toprule
Phase & Lead & Steps & More Information \\
\midrule
Draft & Author & Prepare document in \LaTeX\ using the NREL class file & Section \ref{sec:nrel.cls} \\
 & & Prepare PDF & Section \ref{sec:PDFprep} \\
 & & Converts the tex document to RTF using \texttt{latex2rtf} & Chapter \ref{sec:latextortf}\\
 & & Archive all files, including:
\begin{itemize}  
 \item tex source
 \item images
 \item PDF 
 \end{itemize} & \\
Review & Communications & Review the structure of the PDF & \\
 & & Edit the supplied .DOC or .DOCX file using track changes & \\
Revision & Author & Implements required changes in the \LaTeX\ files. & \\
Publish & Publications & combine the PDF with the appropriate cover sheet(s) & \\
\bottomrule
\end{tabular*}
\end{table}

\section{The NREL \LaTeX\ style file\label{sec:nrel.cls}}
A \LaTeX class called \emph{nrel.cls} has been written that implements the NREL formatting requirements in \LaTeX.

\subsection{Getting \emph{nrel.cls}}
The current version of \emph{nrel.cls} can be downloaded from \url{https://wind-dev.nrel.gov/svn/latex_editing/branches/nrel_class_template/}. Users will be prompted to provide their NREL network username and password.

\subsection{Installing \emph{nrel.cls}}
Any of the following methods can be used to install the class files on a computer.
\begin{description}
\item[All platforms] place the \emph{nrel.cls} and \emph{nrel.bst} in the directory that the \LaTeX\ files are in. This will make the files available to that project, only.
\item[Mac / OSX] place the \emph{nrel.cls} and \emph{nrel.bst} in the user's library in the TexMF tree. For example, place these two files in \emph{~/Library/texmf/tex/latex/nrel}. This will make these files available to all projects.
\item[Windows]
\item[Online, colalborative tools] will require that \emph{nrel.cls} and \emph{nrel.bst} are uploaded to that website.
\end{description}

\subsection{Using \emph{nrel.cls}}
To use the class file, insert the following text in the preamble:

\begin{verbatim}
% ---------------
% PREAMBLE
% ---------------
\newif\iflatextortf

\iflatextortf
 % tell latex2tortf if this is an article or report
 \documentclass[10pt,letterpaper]{report}
 \input{NRELLatex2rtf.tex}
\else
 \documentclass[draft,report]{nrel} 
\fi
\end{verbatim}

This tells \LaTeX to use the correct class file, and defines a set of commands that will be used by \emph{latextortf} to properly convert the latex to a rich text document for reviewing (see Chapter \ref{sec:latextortf}).

\subsection{Options in \emph{nrel.cls}\label{sec:nrel.cls.options}}
The line

\begin{verbatim}
\documentclass[draft,report]{nrel}
\end{verbatim}

\noindent specifies the options (inside the square brackets) that will be passed to the \emph{nrel} class. The options include:
\begin{description}
\item[book]{compile the document using the \LaTeX\ \emph{book} class. This is intended for longer documents and allows the use of chapters.}
\item[report]{compile the document using the \LaTeX\ \emph{report} class. This is intended for longer documents and allows the use of chapters.}
\item[article]{compile the document using the \LaTeX\ \emph{article} class. This is intended for shorter documents such as journal articles. This class does not support the use of chapters.}
\item[memoir]{compile the document using the \LaTeX\ \emph{memoir} class. This option is not recommended because of the challenge with later converting to RTF format for communications review.}
\item[draft]{add a `draft' watermark to all pages and colours all links in blue.}
\item[10pt, 12pt]{set the font size accordingly. The default is 12 point.}
\item[letterpaper, a4paper]{set the paper size. the default is letter paper.}
\end{description}

\subsection{Classes and packages in \emph{nrel.cls}}
\emph{nrel.cls} calls a variety of other packages. Packages are codes that modify the appearance or behaviour of \LaTeX\ to achieve something. Table \ref{Tab:Packages} lists the packages that are explicitly called by \emph{nrel.cls} in the order they are called in. These packages often call other packages, so this is not an exhaustive list.

\begin{table}[!h]
\centering
\caption[Packages supported by the \emph{nrel.cls} class]{Packages supported by the \emph{nrel.cls} class. Unless otherwise stated, packages are not supported by \texttt{latex2rtf}.}
\label{Tab:Packages}
\begin{tabular*}{\textwidth}{p{0.125\textwidth}p{0.1\textwidth}p{0.4\textwidth}p{0.25\textwidth}}
\toprule
Packages & options & functionality & \texttt{latex2rtf} support \\
\midrule
nag & & checks that packages are up to date and looks for bad habits in \LaTeX\ code. & \\
geometry & & sets page size and margins & \checkmark\\
mathptmx& & changes fonts & \\
helvet& & changes fonts & \\
courier& & changes fonts & \\
amsfonts, amssymb & & supplies the AMS fonts, which are useful for mathematics & \\
booktabs & & & \\
graphicx & &graphics handling, including \emph{.eps} figures & \checkmark\\
natbib & sort &handles citations and allows the \verb+\cite+, \verb+\citep+ and \verb+\citet+ citation commands (see Section \ref{Sec:Bib}). & \checkmark\\
fontenc & T1 & &\\
xcolor & & &\\
babel & english & &\\
subfig & & provides the \texttt{subfloat} environment to produce sub figures & \checkmark (\emph{subfloat} is mapped to the \emph{subfigure} command) \\
hyphenat & & &\\
setspace & & &\\
parskip & & &\\
toclof & subfigure & & \\
toclifbind & nottoc, notlot, notlof & &\\
todonotes & & inline and margin to-do notes & \checkmark (`to do' is prefaced with \textbf{To Do:} in the output)\\
caption & & &\\
pdfcomment & & tool-tips. Also calls the package \texttt{hyper ref} & \checkmark (the tool tip is suppressed) \\
\bottomrule
\end{tabular*}
\end{table}

\chapter{Some LaTeX examples}
This chapter includes examples of how to do common tasks using \LaTeX{}. Although most users will be familiar with these commands and environments, these serve as a) a test of the class file and conversion process, and b) examples that are known to work with the class and conversion process. So, when all else fails, users can copy these examples and tailor them to their particular case.

\section{Page numbering}
NREL's convention is to have Roman numerals in the front matter, and then arabic numerals in the main matter of the document (after the tables of contents, figures and tables). This is implemented using the following structure in the document:

\begin{verbatim}
\begin{document}

\maketitle
\pagenumbering{roman}
\setcounter{page}{3}

\renewcommand{\contentsname}{Table of Contents}
\tableofcontents
\clearpage
\listoffigures
\listoftables
\clearpage
\pagenumbering{arabic}
\end{verbatim}

\section{Headings}
\LaTeX{} allows a very simple definition of the document's structure. This document has the following structure:
\begin{itemize}
\item Chapter 1: what is \LaTeX?
\begin{itemize}
\item Section 1: Headings
\item Section 2: Floats
\item Section 3: Mathematics
\item Section 4: Lists
\end{itemize}
\item etc. \ldots
\end{itemize}

\subsection{Chapter}
To define a new chapter, simply write \verb+\chapter{What is \LaTeX?}+.

To use chapters, pass the \texttt{memoir}, \texttt{book}, or \texttt{report} option to \emph{nrel.cls} (see Section \ref{sec:nrel.cls.options}).

\subsection{Sections}
If Chapters are the highest level headings in a document, sections come next, followed by subsections. Although there don't have to be chapters in a document, a \LaTeX\ document does need to have Sections.

So: 

\begin{verbatim}
\section{Headings}
\LaTeX{} allows a very simple definition of the document's structure. 
This document has the following structure:
...
\subsection{Chapter}

\end{verbatim}

\section{Body text}
Body text does not need to be specially identified in \LaTeX{}. Non-printing comments are identified in the source document(s) using the \% symbol.

\section{Mathematics}

\LaTeX is great at typesetting mathematics. The following example is taken from the \url{www.writelatex.com} website:

\begin{quote}
Making inline equations is easy. Let $X_1, X_2, \ldots, X_n$ be a sequence of independent and identically distributed random variables with $\textrm{E}[X_i] = \mu$ and $\textrm{Var}[X_i] = \sigma^2 < \infty$, and let
$$S_n = \frac{X_1 + X_2 + \cdots + X_n}{n}
 = \frac{1}{n}\sum_{i}^{n} X_i$$
denote their mean. Then as $n$ approaches infinity, the random variables $\sqrt{n}(S_n - \mu)$ converge in distribution to a normal $\mathcal{N}(0, \sigma^2)$.
\end{quote}

Alternatively, if numbered equations are required, use the \texttt{equation} environment. For example:

\begin{verbatim}
\begin{equation}
y = mx +c \textrm{.}
\label{eqn:line}
\end{equation}
\end{verbatim}

would give:

\begin{equation}
y = mx+c \textrm{.}
\label{eqn:line}
\end{equation}

\section{Cross references}
Use \texttt{label}s and \texttt{ref}erences to refer back and forth to figures, equations, tables and sections. For example, \verb+Eqn. \ref{eqn:line}+ givess a reference to Eqn. \ref{eqn:line}.

\section{Floats}
Floats are images, tables or other pieces of the document that are free to move to the best place in the document for them. Literally, they `float'. The two most common floats are the tabular environment (for tables) and the figure environment for figures.

\subsection{Tables}
Use the \texttt{tabular} environment to produce basic tables. Table~\ref{tab:widgets} is produced using this code: 

\begin{verbatim}
\begin{table}[!h]
\centering
\caption{\label{tab:widgets}An example table.}
\begin{tabular}{l|r}
Item & Quantity \\\hline
Widgets & 42 \\
Gadgets & 13
\end{tabular}
\end{table}
\end{verbatim}

\begin{table}[!h]
\centering
\caption{\label{tab:widgets}An example table.}
\begin{tabular}{l|r}
Item & Quantity \\\hline
Widgets & 42 \\
Gadgets & 13
\end{tabular}
\end{table}

Resist the temptation to stop table rows early. If all of the delimiters  (\&) are included in each row, the table will be complete and will better translate to RTF later.

\subsection{Figures}
To include a figure in a document, use the \texttt{figure} environment and the \texttt{includegraphics} command.

\begin{verbatim}
\begin{figure}
\includegraphics[width=\textwidth]{figure's-file-name}
\caption{\label{fig:figuresLabel}Caption goes here.}
\end{figure}
\end{verbatim}

\subsection{Subfigures}
Subfigures are implemented using the \texttt{subfig} package. Although this package is deprecated (apparently \texttt{subcaption} is now the preferred package), it plays fairly nicely with \texttt{latex2rtf} so will be used for the foreseeable future. 

The \texttt{label}s in the example below allow us to make references using the \texttt{ref} command, both to the overall figure (Figure \ref{fig:NRELimages}) and the subfigures (Figures \ref{fig:21206} and \ref{fig:20018}) directly. Unfortunately, \texttt{latex2rtf} does not allow multiple \texttt{label}s in a Figure environment, and so only the first label will be kept: therefore, it's best to just use a single label in any one \texttt{figure} environment.

\begin{verbatim}
\begin{figure}
\centering
\hfill
\subfloat[Wind turbines at the Forward Wind Energy Center in Fond du Lac 
 and Dodge Counties, Wisconsin. (Photo by Ruth Baranowski / NREL)
 \label{fig:21206}]{\includegraphics[height=2.5in]{21206}}
\hfill 
\subfloat[Aerial view of the National Wind Technology Center. 
 (Photo by Dennis Schroeder / NREL)\label{fig:20018}]
 {\includegraphics[height=2.5in]{20018}}
\hfill
\caption{NREL images}\label{fig:NRELimages}
\end{figure}
\end{verbatim}
 
\begin{figure}[!h]
\centering
\hfill
\subfloat[Wind turbines at the Forward Wind Energy Center in Fond du Lac and Dodge Counties, Wisconsin. (Photo by Ruth Baranowski / NREL)\label{fig:21206}]{\includegraphics[height=2.5in]{21206}}
~ %add desired spacing between images, e. g. ~, \quad, \qquad etc. (or a blank line to force the subfigure onto a new line)
\hfill
\subfloat[Aerial view of the National Wind Technology Center. (Photo by Dennis Schroeder / NREL)\label{fig:20018}]{\includegraphics[height=2.5in]{20018}}
\hfill
\caption{NREL images}\label{fig:NRELimages}
\end{figure}

If a subfigure is split over two lines using \verb+\\+, make sure those symbols are on their own line.

\section{Lists}

To make lists with automatic numbering, use the \texttt{enumerate} environment:

\begin{enumerate}
\item Like this,
\item and like this.
\end{enumerate}
\dots or bullet points \dots
\begin{itemize}
\item Like this,
\item and like this.
\end{itemize}

\section{Computer code}
Use the \texttt{verbatim} environment to display code. Alternatives like \texttt{lstlisting} do not work. I've tried.

\section{Creating a file structure}
\label{sec:FileStructure}
Use the \texttt{input} command to import other files into your main file. 

\begin{verbatim}
...
% content
\input{NRELRequirements}
\input{LatexAtNREL}
\chapter{Some LaTeX examples}
This chapter includes examples of how to do common tasks using LaTeX{}. Although most users will be familiar with these commands and environments, these serve as a) a test of the class file and conversion process, and b) examples that are known to work with the class and conversion process. So, when all else fails, users can copy these examples and tailor them to their particular case.

Some of the material in this section is based on introductory material from wikibooks, sharelatex.com, and other sources. This should be reviewed before final publication.

\section{Headings}
LaTeX{} allows a very simple definition of the document's structure. This document has the following structure:
\begin{itemize}
\item Chapter 1: what is LaTeX?
\begin{itemize}
\item Section 1: Headings
\item Section 2: Floats
\item Section 3: Mathematics
\item Section 4: Lists
\end{itemize}
\item etc. \ldots
\end{itemize}

\subsection{Chapter}
To define a new chapter, simply write \verb+\chapter{What is LaTeX?}+.

To use chapters, pass the \texttt{memoir}, \texttt{book}, or \texttt{report} option to \emph{nrel.cls} (see Section \ref{sec:nrel.cls.options}).

\subsection{Sections}
If Chapters are the highest level headings in a document, sections come next, followed by subsections. Although there don't have to be chapters in a document, a LaTeX document does need to have Sections.

So: 

\begin{verbatim}
\section{Headings}
LaTeX{} allows a very simple definition of the document's structure. 
This document has the following structure:
...
\subsection{Chapter}

\end{verbatim}

\section{Body text}
Body text does not need to be specially identified in LaTeX{}. Non-printing comments are identified in the source document(s) using the \% symbol.

\section{Mathematics}

LaTeX is great at typesetting mathematics. The following example is taken from the \href{www.writelatex.com}{www.writelatex.com} website:

\begin{quote}
Making inline equations is easy. Let $X_1, X_2, \ldots, X_n$ be a sequence of independent and identically distributed random variables with $\textrm{E}[X_i] = \mu$ and $\textrm{Var}[X_i] = \sigma^2 < \infty$, and let
$$S_n = \frac{X_1 + X_2 + \cdots + X_n}{n}
 = \frac{1}{n}\sum_{i}^{n} X_i$$
denote their mean. Then as $n$ approaches infinity, the random variables $\sqrt{n}(S_n - \mu)$ converge in distribution to a normal $\mathcal{N}(0, \sigma^2)$.
\end{quote}

Alternatively, if numbered equations are required, use the \texttt{equation} environment. For example:

\begin{verbatim}
\begin{equation}
y = mx +c \textrm{.}
\label{eqn:line}
\end{equation}
\end{verbatim}

would give:

\begin{equation}
y = mx+c \textrm{.}
\label{eqn:line}
\end{equation}

\section{Cross references}
Use labels and references to refer back and forth to figures, equations, tables and sections. For example, \verb+Eqn. \ref{eqn:line}+ gives a reference to Eqn. \ref{eqn:line}.

\section{Floats}
Floats are images, tables or other pieces of the document that are free to move to the best place in the document for them. Literally, they `float'. The two most common floats are the tabular environment (for tables) and the figure environment for figures.

\subsection{Tables}
Use the \texttt{tabular} environment to produce basic tables. Table~\ref{tab:widgets} is produced using this code: 

\begin{verbatim}
\begin{table}[!h]
\centering
\caption{An example table.}\label{tab:widgets}
\begin{tabular}{lr}
Item & Quantity \\\hline
Widgets & 42 \\
Gadgets & 13
\end{tabular}
\end{table}
\end{verbatim}

\begin{table}[!h]
\centering
\caption{An example table.}\label{tab:widgets}
\begin{tabular}{lr}
Item & Quantity \\\hline
Widgets & 42 \\
Gadgets & 13
\end{tabular}
\end{table}

Resist the temptation to stop table rows early. If all of the delimiters  (\&) are included in each row, the table will be complete and will better translate to RTF later.

\subsection{Figures}
To include a figure in a document, use the \texttt{figure} environment and the \texttt{includegraphics} command.

\begin{verbatim}
\begin{figure}
\includegraphics[width=\textwidth]{figure's-file-name}
\caption{Caption goes here.}\label{fig:figuresLabel}
\end{figure}
\end{verbatim}

\subsection{Subfigures}
Subfigures are implemented using the \texttt{subfig} package. Although this package is deprecated (apparently \texttt{subcaption} is now the preferred package), it plays fairly nicely with \texttt{latex2rtf} so will be used for the foreseeable future. 

The \texttt{label}s in the example below allow us to make references using the \texttt{ref} command, both to the overall figure (Figure \ref{fig:NRELimages}) and the subfigures (Figures \ref{fig:21206} and \ref{fig:20018}) directly. Unfortunately, \texttt{latex2rtf} does not allow multiple \texttt{label}s in a Figure environment, and so only the first label will be kept: therefore, it's best to just use a single label in any one \texttt{figure} environment.

\begin{verbatim}
\begin{figure}
\centering
\hfill
\subfigure[Wind turbines at the Forward Wind Energy Center in Fond du Lac 
 and Dodge Counties, Wisconsin. (Photo by Ruth Baranowski / NREL)
 \label{fig:21206}]{\includegraphics[height=2.5in]{files/21206}}
\hfill 
\subfigure[Aerial view of the National Wind Technology Center. 
 (Photo by Dennis Schroeder / NREL)\label{fig:20018}]
 {\includegraphics[height=2.5in]{files/20018}}
\hfill
\caption{NREL images}\label{fig:NRELimages}
\end{figure}
\end{verbatim}
 
\begin{figure*}[htp]
\centering
\hfill
\subfigure[Wind turbines at the Forward Wind Energy Center in Fond du Lac and Dodge Counties, Wisconsin. (Photo by Ruth Baranowski / NREL)\label{fig:21206}]{\includegraphics[height=2.5in]{files/21206}}
~ %add desired spacing between images, e. g. ~, \quad, \qquad etc. (or a blank line to force the subfigure onto a new line)
\hfill
\subfigure[Aerial view of the National Wind Technology Center. (Photo by Dennis Schroeder / NREL)\label{fig:20018}]{\includegraphics[height=2.5in]{files/20018}}
\hfill
\caption{NREL images}\label{fig:NRELimages}
\end{figure*}

If a subfigure is split over two lines using \verb+\\+, make sure those symbols are on their own line.

\section{Lists}

To make lists with automatic numbering, use the \texttt{enumerate} environment:

\begin{enumerate}
\item Like this,
\item and like this.
\end{enumerate}
\dots or bullet points \dots
\begin{itemize}
\item Like this,
\item and like this.
\end{itemize}

\section{Computer code}
The \texttt{lstlisting} package has been loaded.

\section{Creating a file structure}
\label{sec:FileStructure}
Use the \texttt{input} command to import other files into your main file. For example, each of the chapters in this report could be in separate files, called \emph{NRELRequirements} (Chapter 1), \emph{LatexAtNREL} (Chapter 2), and so-on. 

\begin{verbatim}
...
% content
\input{NRELRequirements}
\input{LatexAtNREL}
\chapter{Some LaTeX examples}
This chapter includes examples of how to do common tasks using LaTeX{}. Although most users will be familiar with these commands and environments, these serve as a) a test of the class file and conversion process, and b) examples that are known to work with the class and conversion process. So, when all else fails, users can copy these examples and tailor them to their particular case.

Some of the material in this section is based on introductory material from wikibooks, sharelatex.com, and other sources. This should be reviewed before final publication.

\section{Headings}
LaTeX{} allows a very simple definition of the document's structure. This document has the following structure:
\begin{itemize}
\item Chapter 1: what is LaTeX?
\begin{itemize}
\item Section 1: Headings
\item Section 2: Floats
\item Section 3: Mathematics
\item Section 4: Lists
\end{itemize}
\item etc. \ldots
\end{itemize}

\subsection{Chapter}
To define a new chapter, simply write \verb+\chapter{What is LaTeX?}+.

To use chapters, pass the \texttt{memoir}, \texttt{book}, or \texttt{report} option to \emph{nrel.cls} (see Section \ref{sec:nrel.cls.options}).

\subsection{Sections}
If Chapters are the highest level headings in a document, sections come next, followed by subsections. Although there don't have to be chapters in a document, a LaTeX document does need to have Sections.

So: 

\begin{verbatim}
\section{Headings}
LaTeX{} allows a very simple definition of the document's structure. 
This document has the following structure:
...
\subsection{Chapter}

\end{verbatim}

\section{Body text}
Body text does not need to be specially identified in LaTeX{}. Non-printing comments are identified in the source document(s) using the \% symbol.

\section{Mathematics}

LaTeX is great at typesetting mathematics. The following example is taken from the \href{www.writelatex.com}{www.writelatex.com} website:

\begin{quote}
Making inline equations is easy. Let $X_1, X_2, \ldots, X_n$ be a sequence of independent and identically distributed random variables with $\textrm{E}[X_i] = \mu$ and $\textrm{Var}[X_i] = \sigma^2 < \infty$, and let
$$S_n = \frac{X_1 + X_2 + \cdots + X_n}{n}
 = \frac{1}{n}\sum_{i}^{n} X_i$$
denote their mean. Then as $n$ approaches infinity, the random variables $\sqrt{n}(S_n - \mu)$ converge in distribution to a normal $\mathcal{N}(0, \sigma^2)$.
\end{quote}

Alternatively, if numbered equations are required, use the \texttt{equation} environment. For example:

\begin{verbatim}
\begin{equation}
y = mx +c \textrm{.}
\label{eqn:line}
\end{equation}
\end{verbatim}

would give:

\begin{equation}
y = mx+c \textrm{.}
\label{eqn:line}
\end{equation}

\section{Cross references}
Use labels and references to refer back and forth to figures, equations, tables and sections. For example, \verb+Eqn. \ref{eqn:line}+ gives a reference to Eqn. \ref{eqn:line}.

\section{Floats}
Floats are images, tables or other pieces of the document that are free to move to the best place in the document for them. Literally, they `float'. The two most common floats are the tabular environment (for tables) and the figure environment for figures.

\subsection{Tables}
Use the \texttt{tabular} environment to produce basic tables. Table~\ref{tab:widgets} is produced using this code: 

\begin{verbatim}
\begin{table}[!h]
\centering
\caption{An example table.}\label{tab:widgets}
\begin{tabular}{lr}
Item & Quantity \\\hline
Widgets & 42 \\
Gadgets & 13
\end{tabular}
\end{table}
\end{verbatim}

\begin{table}[!h]
\centering
\caption{An example table.}\label{tab:widgets}
\begin{tabular}{lr}
Item & Quantity \\\hline
Widgets & 42 \\
Gadgets & 13
\end{tabular}
\end{table}

Resist the temptation to stop table rows early. If all of the delimiters  (\&) are included in each row, the table will be complete and will better translate to RTF later.

\subsection{Figures}
To include a figure in a document, use the \texttt{figure} environment and the \texttt{includegraphics} command.

\begin{verbatim}
\begin{figure}
\includegraphics[width=\textwidth]{figure's-file-name}
\caption{Caption goes here.}\label{fig:figuresLabel}
\end{figure}
\end{verbatim}

\subsection{Subfigures}
Subfigures are implemented using the \texttt{subfig} package. Although this package is deprecated (apparently \texttt{subcaption} is now the preferred package), it plays fairly nicely with \texttt{latex2rtf} so will be used for the foreseeable future. 

The \texttt{label}s in the example below allow us to make references using the \texttt{ref} command, both to the overall figure (Figure \ref{fig:NRELimages}) and the subfigures (Figures \ref{fig:21206} and \ref{fig:20018}) directly. Unfortunately, \texttt{latex2rtf} does not allow multiple \texttt{label}s in a Figure environment, and so only the first label will be kept: therefore, it's best to just use a single label in any one \texttt{figure} environment.

\begin{verbatim}
\begin{figure}
\centering
\hfill
\subfigure[Wind turbines at the Forward Wind Energy Center in Fond du Lac 
 and Dodge Counties, Wisconsin. (Photo by Ruth Baranowski / NREL)
 \label{fig:21206}]{\includegraphics[height=2.5in]{files/21206}}
\hfill 
\subfigure[Aerial view of the National Wind Technology Center. 
 (Photo by Dennis Schroeder / NREL)\label{fig:20018}]
 {\includegraphics[height=2.5in]{files/20018}}
\hfill
\caption{NREL images}\label{fig:NRELimages}
\end{figure}
\end{verbatim}
 
\begin{figure*}[htp]
\centering
\hfill
\subfigure[Wind turbines at the Forward Wind Energy Center in Fond du Lac and Dodge Counties, Wisconsin. (Photo by Ruth Baranowski / NREL)\label{fig:21206}]{\includegraphics[height=2.5in]{files/21206}}
~ %add desired spacing between images, e. g. ~, \quad, \qquad etc. (or a blank line to force the subfigure onto a new line)
\hfill
\subfigure[Aerial view of the National Wind Technology Center. (Photo by Dennis Schroeder / NREL)\label{fig:20018}]{\includegraphics[height=2.5in]{files/20018}}
\hfill
\caption{NREL images}\label{fig:NRELimages}
\end{figure*}

If a subfigure is split over two lines using \verb+\\+, make sure those symbols are on their own line.

\section{Lists}

To make lists with automatic numbering, use the \texttt{enumerate} environment:

\begin{enumerate}
\item Like this,
\item and like this.
\end{enumerate}
\dots or bullet points \dots
\begin{itemize}
\item Like this,
\item and like this.
\end{itemize}

\section{Computer code}
The \texttt{lstlisting} package has been loaded.

\section{Creating a file structure}
\label{sec:FileStructure}
Use the \texttt{input} command to import other files into your main file. For example, each of the chapters in this report could be in separate files, called \emph{NRELRequirements} (Chapter 1), \emph{LatexAtNREL} (Chapter 2), and so-on. 

\begin{verbatim}
...
% content
\input{NRELRequirements}
\input{LatexAtNREL}
\chapter{Some LaTeX examples}
This chapter includes examples of how to do common tasks using LaTeX{}. Although most users will be familiar with these commands and environments, these serve as a) a test of the class file and conversion process, and b) examples that are known to work with the class and conversion process. So, when all else fails, users can copy these examples and tailor them to their particular case.

Some of the material in this section is based on introductory material from wikibooks, sharelatex.com, and other sources. This should be reviewed before final publication.

\section{Headings}
LaTeX{} allows a very simple definition of the document's structure. This document has the following structure:
\begin{itemize}
\item Chapter 1: what is LaTeX?
\begin{itemize}
\item Section 1: Headings
\item Section 2: Floats
\item Section 3: Mathematics
\item Section 4: Lists
\end{itemize}
\item etc. \ldots
\end{itemize}

\subsection{Chapter}
To define a new chapter, simply write \verb+\chapter{What is LaTeX?}+.

To use chapters, pass the \texttt{memoir}, \texttt{book}, or \texttt{report} option to \emph{nrel.cls} (see Section \ref{sec:nrel.cls.options}).

\subsection{Sections}
If Chapters are the highest level headings in a document, sections come next, followed by subsections. Although there don't have to be chapters in a document, a LaTeX document does need to have Sections.

So: 

\begin{verbatim}
\section{Headings}
LaTeX{} allows a very simple definition of the document's structure. 
This document has the following structure:
...
\subsection{Chapter}

\end{verbatim}

\section{Body text}
Body text does not need to be specially identified in LaTeX{}. Non-printing comments are identified in the source document(s) using the \% symbol.

\section{Mathematics}

LaTeX is great at typesetting mathematics. The following example is taken from the \href{www.writelatex.com}{www.writelatex.com} website:

\begin{quote}
Making inline equations is easy. Let $X_1, X_2, \ldots, X_n$ be a sequence of independent and identically distributed random variables with $\textrm{E}[X_i] = \mu$ and $\textrm{Var}[X_i] = \sigma^2 < \infty$, and let
$$S_n = \frac{X_1 + X_2 + \cdots + X_n}{n}
 = \frac{1}{n}\sum_{i}^{n} X_i$$
denote their mean. Then as $n$ approaches infinity, the random variables $\sqrt{n}(S_n - \mu)$ converge in distribution to a normal $\mathcal{N}(0, \sigma^2)$.
\end{quote}

Alternatively, if numbered equations are required, use the \texttt{equation} environment. For example:

\begin{verbatim}
\begin{equation}
y = mx +c \textrm{.}
\label{eqn:line}
\end{equation}
\end{verbatim}

would give:

\begin{equation}
y = mx+c \textrm{.}
\label{eqn:line}
\end{equation}

\section{Cross references}
Use labels and references to refer back and forth to figures, equations, tables and sections. For example, \verb+Eqn. \ref{eqn:line}+ gives a reference to Eqn. \ref{eqn:line}.

\section{Floats}
Floats are images, tables or other pieces of the document that are free to move to the best place in the document for them. Literally, they `float'. The two most common floats are the tabular environment (for tables) and the figure environment for figures.

\subsection{Tables}
Use the \texttt{tabular} environment to produce basic tables. Table~\ref{tab:widgets} is produced using this code: 

\begin{verbatim}
\begin{table}[!h]
\centering
\caption{An example table.}\label{tab:widgets}
\begin{tabular}{lr}
Item & Quantity \\\hline
Widgets & 42 \\
Gadgets & 13
\end{tabular}
\end{table}
\end{verbatim}

\begin{table}[!h]
\centering
\caption{An example table.}\label{tab:widgets}
\begin{tabular}{lr}
Item & Quantity \\\hline
Widgets & 42 \\
Gadgets & 13
\end{tabular}
\end{table}

Resist the temptation to stop table rows early. If all of the delimiters  (\&) are included in each row, the table will be complete and will better translate to RTF later.

\subsection{Figures}
To include a figure in a document, use the \texttt{figure} environment and the \texttt{includegraphics} command.

\begin{verbatim}
\begin{figure}
\includegraphics[width=\textwidth]{figure's-file-name}
\caption{Caption goes here.}\label{fig:figuresLabel}
\end{figure}
\end{verbatim}

\subsection{Subfigures}
Subfigures are implemented using the \texttt{subfig} package. Although this package is deprecated (apparently \texttt{subcaption} is now the preferred package), it plays fairly nicely with \texttt{latex2rtf} so will be used for the foreseeable future. 

The \texttt{label}s in the example below allow us to make references using the \texttt{ref} command, both to the overall figure (Figure \ref{fig:NRELimages}) and the subfigures (Figures \ref{fig:21206} and \ref{fig:20018}) directly. Unfortunately, \texttt{latex2rtf} does not allow multiple \texttt{label}s in a Figure environment, and so only the first label will be kept: therefore, it's best to just use a single label in any one \texttt{figure} environment.

\begin{verbatim}
\begin{figure}
\centering
\hfill
\subfigure[Wind turbines at the Forward Wind Energy Center in Fond du Lac 
 and Dodge Counties, Wisconsin. (Photo by Ruth Baranowski / NREL)
 \label{fig:21206}]{\includegraphics[height=2.5in]{files/21206}}
\hfill 
\subfigure[Aerial view of the National Wind Technology Center. 
 (Photo by Dennis Schroeder / NREL)\label{fig:20018}]
 {\includegraphics[height=2.5in]{files/20018}}
\hfill
\caption{NREL images}\label{fig:NRELimages}
\end{figure}
\end{verbatim}
 
\begin{figure*}[htp]
\centering
\hfill
\subfigure[Wind turbines at the Forward Wind Energy Center in Fond du Lac and Dodge Counties, Wisconsin. (Photo by Ruth Baranowski / NREL)\label{fig:21206}]{\includegraphics[height=2.5in]{files/21206}}
~ %add desired spacing between images, e. g. ~, \quad, \qquad etc. (or a blank line to force the subfigure onto a new line)
\hfill
\subfigure[Aerial view of the National Wind Technology Center. (Photo by Dennis Schroeder / NREL)\label{fig:20018}]{\includegraphics[height=2.5in]{files/20018}}
\hfill
\caption{NREL images}\label{fig:NRELimages}
\end{figure*}

If a subfigure is split over two lines using \verb+\\+, make sure those symbols are on their own line.

\section{Lists}

To make lists with automatic numbering, use the \texttt{enumerate} environment:

\begin{enumerate}
\item Like this,
\item and like this.
\end{enumerate}
\dots or bullet points \dots
\begin{itemize}
\item Like this,
\item and like this.
\end{itemize}

\section{Computer code}
The \texttt{lstlisting} package has been loaded.

\section{Creating a file structure}
\label{sec:FileStructure}
Use the \texttt{input} command to import other files into your main file. For example, each of the chapters in this report could be in separate files, called \emph{NRELRequirements} (Chapter 1), \emph{LatexAtNREL} (Chapter 2), and so-on. 

\begin{verbatim}
...
% content
\input{NRELRequirements}
\input{LatexAtNREL}
\input{LatexExamples}
\input{ConvertingToDoc}
...
\end{verbatim}

\input{ConvertingToDoc}
...
\end{verbatim}

\input{ConvertingToDoc}
...
\end{verbatim}

\input{ConvertingToWord}
...
\end{verbatim}

\section{Bibliographies and citations}
\label{Sec:Bib}
Always use \texttt{biblatex} to organize references and store them in a single file (e.g. \verb+/Documents/bibliography/bibliography.bib+). Add a bibliography to a document using the following commands:

\begin{verbatim}
\cleardoublepage
\bibliographystyle{nrel}
\label{sec:Bib}
\bibliography{/Users/me/Documents/bibliography/bibliography}
\end{verbatim}

The bibliography will then contain entries with `keys', like \texttt{Lamport\_1986\_a}. Authors can then insert citations to this key throughout their document, using different styles of citation:
\begin{itemize}
\item \verb+\cite{Lamport_1986_a}+ prints a simple \cite{Lamport_1986_a}.
\item \verb+\citep{Lamport_1986_a}+ puts parentheses around it \citep{Lamport_1986_a}.
\item \verb+\citep[e.g][]{Lamport_1986_a}+ puts parentheses around it, and some text in there as well \citep[e.g.][]{Lamport_1986_a}.
\item \verb+\citet{Lamport_1986_a}+ prints it inline, so that according to \citet{Lamport_1986_a}, \ldots.
\end{itemize}

The command \verb+\bibliographystyle{nrel}+ prints a bibliography that follows NREL's style guide. See Chapter \ref{sec:Bib} for an example.

\section{Best practice in writing a document in \LaTeX}
\begin{description}
\item[Create a structure before you get too far.] Authors will find it easier to write documents and make changes if they separate the content of the document from the structure.
\begin{enumerate}
\item Each new \LaTeX\ document should be placed in it's own directory. 
\item Create a main \LaTeX\ file that just contains the preamble, custom commands and uses \texttt{input} to call the content. See Section \ref{sec:FileStructure} for an example where each \texttt{chapter} is contained in its own file. In an article, each \texttt{section} could be contained in its own file.
\item Keep the number of pacakges used to a minimum. If authors feel that something is desperately missing, they can contact the maintainers of the \emph{nrel.cls} file. Not all packages can be used as they lack compatibility.
\end{enumerate}
\item[Focus on content, not appearance.] Don't spend hours trying to adjust fonts, headers or spacing between lines. 
\begin{enumerate}
\item The document produced should meet NREL's requirements if it is compiled using \emph{nrel.cls}. 
\item Don't throw in lots of \texttt{clearpage}s or other commands to push material around. \LaTeX\ is designed to handle that. 
\item Resist the temptation to add or subtract space, change lengths or do other things to modify the layout. 
\item Write!
\end{enumerate}
\end{description}

\chapter{Preparing a .DOC or .DOCX file from \LaTeX \label{sec:latextortf}}
The current preferred method of editing documents that have been produced using \LaTeX\ is to use \texttt{latex2rtf} to convert the tex files to a rich-text format (RTF) document, and then to .DOC or .DOCX format. Communications will edit this document, which could then be used to complete the publications process.

\section{How does \texttt{latex2rtf} work?}
The \texttt{latex2rtf} program reads \LaTeX\ files and converts common \LaTeX\ commands into their RTF equivalent. It is effectively another \LaTeX\ interpreter that knows a limited subset of \LaTeX.

\section{Using \texttt{latex2rtf}}
To convert a document from \LaTeX\ to RTF, follow these steps:
\begin{enumerate}
\item Install \texttt{latex2rtf}, following instructions at \url{http://sourceforge.net/projects/latex2rtf/}
\item Compile the document in \LaTeX using the NREL class with the \texttt{book,report}, or \texttt{article} option, remembering to update the bibliography and cross references. The sequence of commands is:
\begin{enumerate}
\item latex myfile.tex
\item bibtex myfile.tex
\item latex myfile.tex
\item latex myfile.tex
\end{enumerate}
\item Convert the document to RTF format using \texttt{latex2rtf}:
\begin{description}
\item[Mac]{use the terminal to navigate to the directory the files are in. At the prompt, type:

\$> latex2rtf -o outputfilename.rtf -f3 inputfilename.tex}
\item[Windows]
\item[Online/collaborative tools]{Most online tools lack the ability to convert to RTF. Download the \LaTeX\ files and compile them using the usual procedure.}
\end{description}

\item Open the RTF file in Microsoft Word.
\begin{enumerate}
\item If the document contains tables of contents, tables of figures, tables of tables, or cross-references, select that text and update the fields.
\end{enumerate}
\item Save the RTF file as a word-format document.
\end{enumerate}

\section{Using \texttt{latex2rtf} and \LaTeX\ together}
Because \texttt{latex2rtf} only knows a subset of \LaTeX, it is important to account for this when preparing a \LaTeX document. The biggest problem is the lack of many packages, which is why authors are encouraged to use the NREL class file, which is known to work well with \texttt{latex2rtf}. Sometimes, though, it is important to be able to remove formatting for compatibility with \texttt{latex2rtf}, and so the preamble to this document includes a check to see if \texttt{latex2rtf} is being used:

\begin{verbatim}
\newif\iflatextortf
\iflatextortf
	\documentclass[12pt,letterpaper]{report}
	\input{NRELLatex2rtf.tex}
\else
	\documentclass[report]{nrel} 
\fi
\end{verbatim}

If \texttt{latex2rtf} is used, the boolean, \texttt{\textbackslash iflatextortf} will be TRUE and the commands will be interpreted as follows.
\begin{enumerate}
\item Set the document class to a generic \LaTeX{} \emph{article}, \emph{report}, or \emph{book}. 
\item The file \emph{NRELLatex2rtf.tex} will be called, which maps most of the commands that are enabled in \emph{nrel.cls} to simpler versions that can be processed using \texttt{latex2rtf} (see Table \ref{Tab:Packages}).
\end{enumerate}

An example of the content of \emph{NRELLatex2rtf.tex} follows:

\begin{verbatim}
% File NRELLatex2rtf.tex
% set margins
\usepackage[margin=1 in,letterpaper]{geometry}
% use citations
\usepackage[sort]{natbib}
% change the heading of the bibliography
\renewcommand{\bibsection}{\section{References}}
% redefine \pdftooltip so that it behaves differently with and without latextortf
\newcommand{\pdftooltip}[3][]{#2}
%redefine the checkmark
\newcommand{\checkmark}{y\relax}
% redefine booktabs commands
\newcommand{\toprule}{\hline}
\newcommand{\midrule}{\hline}
\newcommand{\bottomrule}{\hline}
% redefine \href
\newcommand{\href}[2]{#1~ (\url{#2})}
% redefine \subfloat to match the \subfigure environment
\usepackage{subfigure}
\makeatletter
\newcommand{\subfloat}[2][]{\subfigure{\textit{Subcaption: \protect{#1}}}{#2}}
%\newcommand{\subfloat}[3][]{\subfigure{#1}{#2}{#3}}
% note that we can only have one '\label' in a figure environment
\makeatother
\newcommand{\subref}[1][]{\ref{#1}}
% redefine \todo so that it gives something useful
\newcommand{\todo}[2][]{\textbf{To Do:}~#2}
% deal with index entries:
\newcommand{\index}[1]{}
\end{verbatim}

Authors are encouraged to copy this approach, adjusting the content of \emph{NRELLatex2rtf.tex} according to their needs. 

\section{Indexes}
Index entries will not be correctly converted to an RTF document. If an index is required in the PDF version, redefine the index command so that it doesn't generate an output during the conversion to RTF. Try

\begin{verbatim}
\iflatextortf
\else
\newcommand{\index}[1]{}
\fi
\end{verbatim}

in the preamble to kill index entries. Alternatively, the \verb+\newcommand+ could go in the  \emph{NRELLatex2rtf.tex} file.

\section{What to do when the conversion to RTF fails}
It is more than likely that the conversion to RTF will fail at some point. There are a few ways to deal with this:

\begin{description}
\item[Convert early and often.] Check that the document converts using \texttt{latex2rtf} every time a new environment is added.
\item[Try section-by-section.] Comment out the majority of the document and try to compile bit-by-bit. This will let you localize the error.
\item[Check new packages.] Please avoid using new packages. If a package has to be used, try the conversion immediately. If \texttt{latex2rtf} doesn't support the package, edit the file \emph{NRELLatex2rtf.tex} to redefine those commands to something that will convert appropriately. Put \emph{NRELLatex2rtf.tex} in the same directory as the \LaTeX\ file to be converted.
\item[Avoid custom commands.] \texttt{latex2rtf} sometimes chokes on custom commands. A list of all recognized commands is available in the manual at \url{http://latex2rtf.sourceforge.net/latex2rtf.pdf}. If custom commands are used, they may need to be redefined to work with the commands that \texttt{latex2rtf} does recognize. This can also be done in \emph{NRELLatex2rtf.tex}. You can check macros using the flag \verb+-d2+ when running \texttt{latex2rtf}.
\item[Use copy-paste.] Compile the whole document as a PDF, and save it somewhere. Then recompile using the reduced document that works with \texttt{latex2rtf}. Edit this in word and copy in the bits that killed the conversion.
\item[Talk to a communications rep.] If a document cannot be produced any other way than \LaTeX\ with lots of packages, and \texttt{latex2rtf} just refuses to process it, discuss the process for having the PDF processed.
\end{description}

\chapter{Preparing a high-quality PDF from LaTeX\label{sec:PDFprep}}
If the author chooses to complete the publications process using \LaTeX\, the author must incorporate feedback and edits in to the \LaTeX\ source files and prepare the final PDF, following these guidelines.

\section{PDF tagging\label{sec:PDFtagging}}
PDF tagging is a process whereby the components of the PDF document (headings, figures, tables, text) are marked so that a document reader can understand the document. This is useful when text to speech converters are being used. The process of tagging is also known as structuring, so that a tagged document might also be referred to as a structured document.

\LaTeX\ does not prepare a tagged PDF document. The current solution to this is to use the tagging capability built in to Adobe's Acrobat Pro.

To prepare a tagged document, follow these steps:
\begin{enumerate}
\item Add tags. Go to the `Advanced' menu. Select `Accessibility', then `Add tags to document'.
\item Add alternative text for figures. Context-click the Figure, select `Properties', and fill in `Alternate Text'. Alternatively, try the process outlined below.
\item Specify the document language. Go to the `File' menu. Select `Document Properties', then the `Advanced' tab, `Language' field. In some versions of Acrobat, the sequence is `File', `Properties', `Reading Options', `Language'.
\item Define tab order.
\begin{enumerate}
\item Go to the `View' menu. Select `Navigation tabs', then `Pages'.
\item Click on any page, then type Ctrl-A (or Command-A on a Mac) to select all the pages.
\item Go to the `Options' menu in the top right of the dialog box, and select `Page Properties'
\item In the `Tab Order' tab, select `Use document structure'.
\end{enumerate}
\item Make sure tables have headings. 
\begin{enumerate}
\item Go to the `View' menu. Select `Navigation tabs', then `Tags'.
\item Select the `Tags' tab. This panel shows the document structure as a tree.
\item Navigate to the table cells that should be headers.
\item Check they have the type <TH>. If not, then right click on the header cell, select `properties', select the `Tag' tab, and change the value for `Type' to <TH>.
\end{enumerate}
\item Make sure all Chapters (or sections, if there are no chapters in the document) are correctly tagged.
\end{enumerate}

\section{Alt-text on images and equations\label{sec:PDFalttext}}
`Alt text' is a textual description of an equation, link or figure. The following short equation should pop-up some text when a user passes a mouse over it. This should work in most PDF readers:
\begin{equation}
\pdftooltip{a^2+b^2=c^2}{An equation}
\end{equation}

The alt text can be added after the PDF is compiled, or written in to the source document. The rest of this section describes how it can be added to the source and generated by \LaTeX\ using the \href{pdfcomment}{http://www.ctan.org/pkg/pdfcomment} package. The general form of the command is:

\begin{verbatim}
\pdftooltip{<item>}{<pop-up text>}
\end{verbatim}

The previous equation was generated using this code:

\begin{verbatim}
\begin{equation}
\pdftooltip{a^2+b^2=c^2}{An equation}
\end{equation}
\end{verbatim}

The same approach can be used to create alt-text for images. For example, Figure \ref{fig:NRELimagesWithAltText} has been labeled. The code for this image is:

\begin{verbatim}
\begin{figure}[!h]
\centering
\hfill
\subfloat[Wind turbines at the Forward Wind Energy Center in Fond du Lac
and Dodge Counties, Wisconsin. (Photo by Ruth Baranowski / NREL)]
{\pdftooltip{\includegraphics[height=2.5in]{21206}}{This is an image}}
~ 
\hfill
\subfloat[Aerial view of the National Wind Technology Center. 
(Photo by Dennis Schroeder / NREL)]
{\includegraphics[height=2.5in]{20018}}
\hfill
\caption{NREL images}\label{fig:NRELimagesWithAltText}
\end{figure}
\end{verbatim}

\begin{figure}[!h]
\centering
\hfill
\subfloat[Wind turbines at the Forward Wind Energy Center in Fond du Lac and Dodge Counties, Wisconsin. (Photo by Ruth Baranowski / NREL)]{\pdftooltip{\includegraphics[height=2.5in]{21206}}{This is an image. It may be possible to propagate the caption into this text.}}
~ %add desired spacing between images, e. g. ~, \quad, \qquad etc. (or a blank line to force the subfigure onto a new line)
\hfill
\subfloat[Aerial view of the National Wind Technology Center. (Photo by Dennis Schroeder / NREL)]{\includegraphics[height=2.5in]{20018}}
\hfill
\caption{NREL images}\label{fig:NRELimagesWithAltText}
\end{figure}

Alt-text is not processed by \texttt{latex2rtf}. So, if the author anticipates finishing the publication solely as a .DOC or .DOCX file, they do not need to use alt-text.

\section{Embedded fonts}
NREL requires that all fonts be embedded in the the final PDF. Check the PDF for embedded fonts using a PDF viewer. For example, in Adobe Acrobat Reader, look at the `fonts' tag of the document properties. If any fonts are not shown as being an \emph{embedded subset}, try the conversion again. 

Encapsulated postscript figures are particularly prone to having undefined fonts. Check by compiling the document in draft mode, and seeing if the fonts are still present in the output PDF. To fix this problem, consider changing the \emph{.eps} file to a \emph{.png}. To do this `on the fly', use this in the document's preamble:

\begin{verbatim}
\usepackage{epstopdf}
\epstopdfDeclareGraphicsRule
 {.eps}{png}{.png}{convert eps:\SourceFile.\SourceExt png:\OutputFile}
\AppendGraphicsExtensions{.png}
\end{verbatim}


% bibliography
\cleardoublepage
\bibliographystyle{nrel}
%\bibintoc
\label{sec:Bib}
\bibliography{bibliography}


\end{document}